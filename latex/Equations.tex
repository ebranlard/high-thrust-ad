\documentclass[11pt]{article}
\usepackage{amsmath}
\usepackage{amssymb}
\usepackage{relsize} % mathsmaller mathlarger

\renewcommand{\d}{\mathrm{d}}
\newcommand{\dr}{{\d}r}
\newcommand{\dA}{{\d}A}
\newcommand{\dQ}{{\d}Q}
\newcommand{\dP}{{\d}P}
\newcommand{\dT}{{\d}T}
\newcommand{\eqdef}{\stackrel{\mathsmaller{\mathsmaller{\mathsmaller{\triangle}}}}{=}} 
\renewcommand{\v}[1]{\underline{#1}}
\newcommand{\CTloc}{\ensuremath{C_{t}}}
\newcommand{\CPloc}{\ensuremath{C_{p}}}
\newcommand{\CQloc}{\ensuremath{C_{q}}}


\begin{document}

\section{Stuff relevant for us for now}
The actuator disk simulation takes as input the loading on each blade and smears it out onto the disk. The following loading is prescribed on each blade:
\begin{align}
\frac{\dT_B(r)}{\dr} = \frac{1}{B}\rho U_0^2 \pi r \,C_t(r) \quad \text{[N/m]}
      %\label{eq:}
\end{align}
where $B$ is the number of blades (3), and $C_t(r)$ is the local thrust coefficient.
% 
The total thrust force on the actuator disk is:
\begin{align}
    T = \int_0^R B\dT_B = \frac{1}{2} \rho U_0^2 2\pi \int_0^r r C_t(r) \dr
\end{align}
The total thrust coefficient is:
\begin{align}
C_{T}&\eqdef \frac{T}{\frac{1}{2} \rho \pi R^2 U_0^2}
= \frac{2}{R^2} \int_0^R  r C_t(r) \dr = 2 \int_0^1 \overline{r} C_t(\overline{r}) {\d}\overline{r}
% 
\end{align}

\begin{align}
C_{T} = 2\int_0^1 \overline{r} C_t(\overline{r}) {\d}\overline{r}
 = 2 C_{T,0}\left[ \int_0^{r_1} \overline{r} \frac{\overline{r}}{r_1}   {\d}\overline{r}
 +\int_{r_1}^{1} \overline{r} 1  {\d}\overline{r} \right] 
 = 2 C_{T,0}\left[1/2-\frac{r_1^2}{2}+\frac{r_1^2}{3} \right]
% 
\end{align}
\begin{align}
 \int_0^1 \overline{r} {\d}\overline{r}
 =1/2
\end{align}
\begin{align}
Ratio =  1-r_1^2\left(\frac{2}{3}-1\right)
      =  1+\frac{r_1^2}{3}
\end{align}


\section{Stuff from my book p117}


The local thrust force $\dT_{B}$ and torque $\dQ_{B}$ from the section $r$ of the blade $B$ is defined as:
%
\begin{align}
    \dT_{B}(r)&\eqdef \v{dF}_B(r)\cdot \v{e}_z\, \dr  \\ %                = \rho \Gamma_B(r) U_t(r) \\
    \dQ_{B}(r)&\eqdef \left[ r\v{e}_r\times \v{dF}_B(r) \dr \right] \cdot \v{e}_z %= \rho \Gamma_B(r) U_n(r) r\dr 
\end{align}
The term \emph{local} is here used to highlight the fact that these quantities are defined for a given radius, as opposed to integrated quantities. The term \emph{elementary} is also used in this book.
%
The local thrust force $\dT(r)$ and torque $\dQ(r)$ on the rotor are obtained by summing the contribution from each blade, i.e. $\dT(r)=\sum_B \dT_B(r)$ and $\dQ(r)=\sum_B \dQ_B(r)$. The elementary power is defined as $\dP=\Omega\,\dQ$. The dimensionless local thrust, torque and power coefficients are defined using the reference speed $U_\text{ref}$ and the area of the elementary annulus $2\pi r \dr$ as follows:
\begin{align}
    {\CTloc}(r)\eqdef\frac{\dT(r)}{\frac{1}{2}\rho U_\text{ref}^2 2\pi r \dr}, \quad
    {\CQloc}(r)\eqdef\frac{\dQ(r)}{\frac{1}{2}\rho U_\text{ref}^2 r 2 \pi r \dr}  ,\quad
    {\CPloc}(r)\eqdef\frac{\dP(r)}{\frac{1}{2}\rho U_\text{ref}^3 2 \pi r \dr} 
    \label{eq:CtCqCpDef}
\end{align}
The total thrust, torque and power are obtained by integration of the elementary quantities along the span of the blade:
\begin{alignat}{3}
    T\eqdef \int_{r_\text{hub}}^R \dT, \quad
    Q\eqdef \int_{r_\text{hub}}^R \dQ, \quad
    P\eqdef \int_{r_\text{hub}}^R \dP
\end{alignat}
The dimensionless coefficients associated to these integrated quantities are:
\begin{alignat}{4}
    C_{T}&\eqdef \frac{T}{\frac{1}{2} \rho A_\text{ref} U_\text{ref}^2}, \quad
    C_{Q}&\eqdef \frac{Q}{\frac{1}{2} \rho A_\text{ref} U_\text{ref}^2 R},\quad
    C_{P}&\eqdef \frac{P}{\frac{1}{2} \rho A_\text{ref} U_\text{ref}^3 } 
    \label{eq:CTCQCPDefBase}
\end{alignat}
where $A_\text{ref}$ is commonly chosen as $A_\text{ref}=\pi R^2$ for wind turbines. The swept area $\pi (R^2-r_\text{hub}^2)$ is a valid choice but not recommended since it is not as widely used.
% 
The above may be re-written as function of the local coefficients as follows:
\begin{alignat}{4}
    C_{T}&\eqdef \frac{2\pi}{A_\text{ref}}\int_{r_\text{hub}}^R r   \CTloc(r) \dr, \quad
    C_{Q}&\eqdef \frac{2\pi}{A_\text{ref}R}\int_{r_\text{hub}}^R r^2 \CQloc(r) \dr,\quad
    C_{P}&\eqdef \frac{2\pi}{A_\text{ref}}\int_{r_\text{hub}}^R r   \CPloc(r) \dr 
    \label{eq:CTCQCPDef}
\end{alignat}
It is noted that by definition the following relations always holds:
\begin{align}
    \CPloc(r) &\equiv \lambda_r \CQloc(r) \label{eq:GeneralRelationsCpCq}\\
    C_P       &\equiv \lambda C_Q \label{eq:GeneralRelationsCPCQ}
\end{align}
%


\end{document}
